\chapter{Multiple Cameras for SLAM}
\label{chapter:MultiCamSLAM}

\section{Motivation}

Talk about the pros and cons of both cameras.  In short

\begin{center}
 \begin{tabular}{ | l | p{5.5cm} | p{5.5cm} | }
  \hline
  \bf Camera & \bf Pros & \bf Cons \\ \hline
  omni 
  & sees 360 degrees, \newline 
  tracks points for longer \newline
  orientation invariant place recognition
  & non metric scale \newline 
  scale drift \newline less robust pose estimation\\ \hline
  
  stereo 
  & 3D data \newline metric scale \newline no scale drift
  & small fov \newline
  more susceptible to occlusions \newline 
  tracks features not as long \newline 
  harder to get an even distribution of features \\ \hline
 \end{tabular}

\end{center}


Therefore combine them to get the ultimate visual slam setup!

\section{Approach}

Different approaches
\begin{itemize}
 \item Do SLAM with omni-cam, fix scale with stereo
 \item Do SLAM with stereo, uses omni to track features for longer, image based loop closure
 \item Tight integration of omni and stereo, doing SLAM over all 3 cameras simultaneously. 
Requires entirely new SLAM pipeline
\end{itemize}

Discuss.  Why did we choose the later and more precisely loop closure.
Give examples.
\\

Maybe state the fact that omni for loop closure is not so straight forward due to non metric scale and the solution will be given later